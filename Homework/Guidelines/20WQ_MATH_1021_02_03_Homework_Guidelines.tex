%\documentclass[10pt]{article}
%
%
%\usepackage{multicol}
%\usepackage{enumitem}
%\setlist{noitemsep}
%
%\usepackage{graphicx}
%\usepackage{amsmath}
%\usepackage{amssymb}
%\usepackage{booktabs}
%\usepackage{multirow}
%%\usepackage[margin=.75in]{geometry}
%\usepackage[margin=.75in]{geometry}
%\usepackage{fancyhdr}
%\pagestyle{fancy}
%\fancyhead{}
%\fancyfoot{}
%\renewcommand{\headrulewidth}{0pt}
%%\fancyfoot[R] {\thepage\ of\ \pageref{lastpage}}
%\fancyfoot[L] {\textbf{These policies are intended as initial guidelines. Some changes may be necessary.}}
%        
%\fancyhead[C]{{\Large \textbf{Homework Guidelines for UCOR 1200
%%{\tiny Section \sectionnumber{}}}}
%{\large Sections 03 \& 07}}}
%}
%
%\begin{document}


\documentclass[10pt]{article}
\input{../homework_header.tex}
\usepackage{enumitem}
\setlist{noitemsep}
\geometry{margin=.75in}

\newcommand{\Year}{2020}
\newcommand{\Qtr}{Winter }
\newcommand{\DueTime}{5:00 PM}
\newcommand{\Class}{MATH 1021-02 \& -03}
\newcommand{\ClassShort}{MATH 1021}
\newcommand{\CourseName}{Precalculus - Algebra}
\newcommand{\FirstDue}{Due: 1/10/2019}

\begin{document}

\rhead{}
\lhead{}
\chead{{\Large \textbf{Homework Guidelines for \Class: \CourseName,  \Qtr \Year}}}
%{\large Sections 01, 04 \& 06}}}}
\lfoot{}

%\begin{center}
%\vspace*{-.5in}
%%{\Large \textbf{Syllabus for MATH 118
%%%{\tiny Section \sectionnumber{}}}}
%%{\large Section 03}}}
%%
%{\large \textbf{Mathematics of Epidemics}}
%\end{center}
%\hrule



\vspace*{-.5in}
\section*{Guidelines:}

Mastery of the material in this course will require a lot of practice in the form of homework.\footnote{A rule of thumb, long espoused by college professors: on average, students need to spend at least two hours working outside of class for every hour spent in class.  Since \ClassShort\ meets for roughly 4.5 hours per week, you should expect to spend at least 9 hours per week studying and preparing for class.}  Assignments will be posted on the Canvas website daily and will be due at \DueTime\ on the posted due date.  It is your responsibility to check the course website  to find the current homework assignment and due date.\\

\noindent The homework that you submit each day is an opportunity to demonstrate your mastery
of the course material. 
Homework solutions involve much more than just the final answer, since a correct final
answer does not necessarily validate all the work that led up to it; just as an incorrect final answer
does not represent an incorrect approach to solving a problem.
Therefore, the solutions you submit %should be more than a sequence of
%calculations and algebraic manipulations followed by a number in a box.  
%  Instead, your solutions
should be an explanation of what you did to solve the problem, why you did it, and how you
did it. As a rule of thumb, another student in a different section of this class should be able to read
and understand your solution without having access to a textbook or solution key.\\




\vspace*{-.2in}
\section*{Policies:}

In order to make life easier for everyone,\footnote{For the purposes of this document, consider ``everyone'' to consist of our grader, me, and you.} please follow the procedures outlined below:  %\textbf{Points may be deducted if you fail to follow these procedures.}

\begin{description}[leftmargin=0in,itemsep=10pt]
    
\item[\textbf{Heading:}]  In the upper right hand corner of every sheet of paper, please include your name, course and section number, the assignment number, as well as the due date.  For example:

\begin{itemize}
	\item[] Your Name
	\item[] \Class
	\item[] Homework \# 1
	\item[] \FirstDue
\end{itemize}

\item[\textbf{Numbering:}] Please list all problems in numerical order.  If problems are assigned from more than one section of the text, please clearly label which section the problem came from.

\item[\textbf{Neatness:}] 
In most other classes, you will type your homework, reports, projects, and labs.  You will not be required to learn how to typeset mathematics for this course, but you will be expected to turn in neat and legible work.  %Your work should be neat and orderly and should not include scratch work, incorrect work, crossed out work, or messy erasing marks.  

The best way to do this is to prepare your math homework like you would a paper in any other class by first creating a draft.  
I suggest that you work the problems on scratch paper, or in a homework notebook.  Once you know your solutions are good, recopy them and make sure they are neat, organized, and presentable.
% and then write up your solutions on a fresh sheet of paper to turn in.  

Please use loose leaf paper or other paper with a straight edge, and please staple the pages together. %Paper torn out of a spiral notebook with ragged edges will not be accepted.  
%All pages must be stapled together. 

\textbf{The grader will be instructed to deduct points for messiness.}

\item[\textbf{Showing Work:}]  
One of the goals of this class is to improve your mathematical communication skills, so please use proper notation at all times. In particular, pay close attention to your use of ``=,'' be sure to show all of your steps, and explain your reasoning when necessary.
%Don't assume that anyone reading your work will know what you mean or consider something obvious. 
Write as if your paper is being read by one of your peers who is taking the same class, but is two or three weeks behind our class material. 

%Write your solutions as if someone who doesn't remember how to do the problemswill be using your work to help understand the material, because when it comes time tostudy for an exam, that someone will be you! Use proper notation at all times. Don't get sloppy or take shortcuts, and pay close attention to your use of ``=.''



%Write as if your paper is being read by one of your peers who is taking the same class, but is
%two or three weeks behind our class material.



%\item[\textbf{Graphs:}]  Any graphs you are asked to sketch should be done on graph paper and clearly numbered.  You may either attach a separate sheet of graph paper at the end of your assignment, or you may cut out and tape/glue the graph next to the rest of the homework problem.  If your graphs appear on a separate sheet, please write something like ``see attached graph'' where the graph would otherwise appear in your homework.

\item[\textbf{Deadlines:}]  \textbf{Homework is due by \DueTime\ on the due date}, unless otherwise specifically mentioned on the assignment.  If the grader has not already collected the homework, you may turn in your homework late.  However, \textbf{once the grader has collected the homework, no late assignments will be accepted.}  There will be an envelope in the hallway near my office (ENGR 409) for turning in your homework.  Please make sure your homework goes in the correct folder so that it does not get lost.

\item[\textbf{Grading Scheme:}]
I will select a few problems from each assignment to be graded for accuracy.  The rest of the homework points will be given for completeness.\footnote{More details will be provided after I have met with our grader to finalize the grading scheme.}  Your two lowest homework scores will be dropped before calculating your grade at the end of the quarter.

%\item[\textbf{Grading Scheme:}]  I will select a few problems from each assignment to be graded for accuracy.\footnote{I will provide mor}  Each graded problem will be worth two points.  You will also receive a score of 0, 1 or 2 for the overall completeness of the assignment. for a total of 10 points for each assignment.  
%Your lowest homework score will be dropped before calculating your grade at the end of the quarter.
    
\label{lastpage}

\end{description}



\end{document}